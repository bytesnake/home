% vim : set tabstop=4 shiftwidth=4 expandtab:

\documentclass[varwidth=15cm, border=.5cm]{standalone}
\usepackage{amsmath, amsfonts, amsthm, hyperref, mdframed, thmtools}

\mdfsetup{linewidth=1.5pt, topline=false, rightline=false, bottomline=false, leftline=true}
\declaretheoremstyle[ notebraces={}{}, headpunct=, headformat=\NAME{} --\NOTE, postheadhook=\leavevmode\begin{mdframed}, prefoothook=\end{mdframed}, ]{default}
\declaretheorem[ name=Satz, style=default,]{theorem}
\declaretheorem[ name=Definition, style=default,]{definition}
\declaretheorem[ name=Remark, style=default,]{remark}
\declaretheorem[ name=Proposition, style=default,]{proposition}

\DeclareMathOperator{\prox}{prox}
\DeclareMathOperator{\argmin}{argmin}
\DeclareMathOperator{\id}{Id}

\newcommand{\innerp}[2]{\left\langle #1 \vert #2 \right\rangle}

\begin{document}

\begin{theorem}[label=pb8y7i3e, name=Boundedness of operators]
	All odd numbers are prime. 

	\begin{theorem}[A side note]
		Blub
	\end{theorem}
\end{theorem}

\begin{definition}[label=f6izn8iw, name=Proximal Operator]
	The proximal operator (also see resolvant for CCP case) is given by 

	\[
        \prox_{\lambda f}(x) = \argmin_y \{ \Theta(y; x) = \lambda f(y) +
        \frac{1}{2}\lVert y - x\rVert_2^2\}
	\]

    \begin{theorem}[label=78req5r7, name=Convergence with Lyapunov Analysis]
        Let \( f\in\mathcal{F}_{0,\infty} \). For any \( k\in\mathbb{N}, A_k, \lambda_k > 0\) and any \( x_k \), the inequality
        \begin{alignat*}{6}
            &A_{k+1}&&(f(x_{k+1}) &&- f(x^\star)) &&+ &&\frac{1}{2}\lVert x_{k+1} &&- x_\star\rVert^2_2  \\
            \leq &A_k&&(f(x_k)   &&- f(x^\star)) &&+ &&\frac{1}{2}\lVert x_k   &&- x^\star\rVert_2^2
        \end{alignat*}

        holds for iteration \( x_{k+1} = \prox_{\lambda_kf}(x_k) \) and number \(A_{k+1}= A_k + \lambda_k\).
    \end{theorem}

    \begin{remark}[label=hnns6j86, name=Computation complexity]
        Proximal operations are in general expensive, sometimes as expensive as
        minimizing the function itself. There are however many instances of
        function \( f \) for which an analytic solution exists. For composite
        problems, those parts are isolated and solved separately by a
        conceptional proximal operator.
    \end{remark}
\end{definition}

\begin{definition}[label=m2qlq7ne, name=Cocoercive Operators]
	An operator \( \mathcal{T} \) is \textit{\( \mu \)-stronly monotone} or \textit{\( \mu \)-cocoercive}  if there exists 
	a \( \beta > 0 \), for which the following inequality holds globally
	\[
		\innerp{x - y}{\mathcal{T}x - \mathcal{T}y} \geq \beta\lVert\mathcal{T}x - \mathcal{T}y\rVert^2 \quad \forall (x, y).
	\]
	\begin{proposition}[label=wk4xp5i4, name=Cocoerciveness of nonexpansive operators]
		Let \( \mathcal{T} \) be a proper operator, then (\( \mathcal{T} \) is non-expansive) 
		\( \iff \) (\( \id - \mathcal{T} \) is \( 1/2\)-cocoercive).
	\end{proposition}

	\begin{proposition}[label=fgebss0l, name=Sum of cocoercive operators
		under transformation]
		Let \( \mathcal{T}_i \) a family of \( \beta_i \)-cocoercive operators with \( L_i \in \mathcal{B}(\mathcal{H}, \mathcal{K}_i) \)
		and \( \beta_i > 0 \). Set

		\[
			T = \sum_i L_i^\star\mathcal{T}_iL_i \quad \beta^{-1} = \sum_i \frac{\lVert L_i\rVert^2}{\beta_i}
		\]

		then \( \mathcal{T} \) is \( \beta \)-cocoercive.
	\end{proposition}

	%\begin{remark}[label=xno_2i29, name=Connection to Lipschitz continuous]
	%\end{remark}
\end{definition}

\end{document}
